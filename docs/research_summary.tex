\documentclass[10pt,a4paper,twocolumn]{article}
\usepackage[margin=0.5in]{geometry}
\usepackage{fontspec}
\setmainfont{Arial}
\usepackage{titlesec}
\usepackage{parskip}
\usepackage{enumitem}

% Compact spacing
\titlespacing*{\section}{0pt}{6pt}{2pt}
\titlespacing*{\subsection}{0pt}{4pt}{2pt}
\setlength{\parskip}{3pt}

\title{\vspace{-1.5cm}\large\textbf{Strategic Context: Roadside Drug Testing in Australia (2016--2025)}\\\small A Strategic Review of Enforcement Paradigms, Legislative Discord, and Safety Efficacy}
\date{}
\author{}

\begin{document}
\maketitle
\vspace{-1.5cm}

\section*{Executive Summary}
The administration of road safety in Australia has historically been defined by deterrence theory, exemplified by Random Breath Testing (RBT). However, 2016--present represents a turbulent epoch for Roadside Drug Testing (RDT). While RBT is scientifically settled, RDT is the locus of tension between technological enforcement, pharmacological realities, and health policies (e.g., medicinal cannabis). 

This review synthesizes data from BITRE, CARRS-Q, MUARC, and BOCSAR. It identifies that while RDT programs have expanded (NSW conducts >150,000 tests/year), the correlation between testing volume and reduced road trauma is statistically ambiguous. Unlike alcohol enforcement, drug enforcement suffers from a "deterrence gap," particularly among high-risk methylamphetamine users. A critical finding is the chasm between "Zero Tolerance" legislation and scientific understanding of impairment; presence-based detection often captures unimpaired drivers (e.g., medicinal cannabis patients) while potentially missing acutely impaired ones.

\section*{Evolution from Alcohol to Illicit Substances}
Australian road safety law relies on the success of alcohol enforcement (0.05\% BAC). RDT, however, lacks a technological impairment scale. Consequently, jurisdictions adopted a "Zero Tolerance" model (e.g., Road Transport Act 2013 NSW), where the offense is the \textit{presence} of THC, Methylamphetamine, or MDMA, not dangerous driving. This decoupled the offense from safety risk.

\section*{The Shift in Trauma Profiles}
Illicit drugs have displaced alcohol as a primary factor in serious crashes. In 2023, drug driving was a factor in 16.8\% of fatal crashes, overtaking drink driving (12.0\%). In Victoria, methylamphetamine presence in injured drivers rose to 9.1\%, suggesting current deterrence fails to influence high-risk stimulant users.

\section*{Risk Hierarchy (Drummer Studies)}
\textbf{Methylamphetamine:} Unequivocal risk. High odds ratio for crash culpability due to aggression/speeding (acute) and fatigue (withdrawal).
\textbf{THC:} Nuanced risk. Drivers with THC $\ge$ 5 ng/mL show increased risk, but low levels often correlate with cautious driving. The "Zero Tolerance" model ignores this dose-dependent reality.

\section*{Technology \& The "Impairment Gap"}
Australian RDT uses oral fluid screening (e.g., Securetec DrugWipe).
\begin{itemize}[leftmargin=*]
    \item \textbf{False Negatives:} High thresholds mean drivers with impairing levels may pass.
    \item \textbf{Passive Residue:} THC remains in the oral cavity long after impairment subsides, triggering positives for unimpaired drivers.
\end{itemize}

\section*{Operational Economics}
RDT is resource-intensive ($20--$40/test vs. cents for RBT). This dictates a "Targeted Risk" strategy rather than RBT's "Anywhere, Anytime" ubiquity, reducing general deterrence.

\section*{Failure of General Deterrence}
BOCSAR (2024) found a "weak and inconsistent" relationship between testing volumes and detection rates. "Punishment avoidance" reinforces recidivism; heavy users often drive hundreds of times without detection.

\section*{Legislative Dissonance}
Since 2016, medicinal cannabis is a valid therapeutic good (Commonwealth) but an illicit poison for driving (State). This "legislative dissonance" forces patients to choose between treatment and mobility, a unique burden not placed on opioid or benzodiazepine users.

\section*{Recommendations}
\begin{enumerate}[leftmargin=*]
    \item \textbf{Adopt Tasmanian Model:} Allow a medical defense for unimpaired cannabis patients.
    \item \textbf{Stratify Penalties:} Distinguish between high-risk methylamphetamine and lower-risk residual THC.
    \item \textbf{Enhance Data:} Move from "process evaluations" (counts) to "outcome evaluations" (safety).
\end{enumerate}

\end{document}
