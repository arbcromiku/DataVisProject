\documentclass[12pt,a4paper]{article}

% Core layout
\usepackage[a4paper,margin=1in]{geometry}
\usepackage{fontspec}
\setmainfont{TeX Gyre Pagella}
\setsansfont{TeX Gyre Heros}
\setmonofont{TeX Gyre Cursor}
\usepackage{setspace}
\onehalfspacing
\hbadness=10000
\hfuzz=1pt

% Figures and tables
\usepackage{xcolor}
\definecolor{primary}{HTML}{2563EB}
\definecolor{accent}{HTML}{16A34A}
\definecolor{ink}{HTML}{111827}
\definecolor{muted}{HTML}{6B7280}
\usepackage{graphicx}
\usepackage{booktabs}
\usepackage{longtable}
\usepackage{tabularx}
\usepackage{array}
\usepackage{float}
\usepackage{subcaption}
\usepackage{caption}
\captionsetup{labelfont=bf,font=small}
\setlength{\tabcolsep}{4pt}
\newcolumntype{Y}{>{\RaggedRight\arraybackslash}X}
\newcolumntype{Z}[1]{>{\RaggedRight\arraybackslash}p{#1}}
\usepackage{amsmath}

% Navigation
\usepackage[colorlinks=true,linkcolor=primary,urlcolor=primary,citecolor=primary]{hyperref}
\usepackage{bookmark}
\usepackage{cleveref}

% Headings
\usepackage{titlesec}
\titleformat{\section}{\Large\bfseries\color{ink}}{\thesection}{0.5em}{}
\titleformat{\subsection}{\large\bfseries\color{ink}}{\thesubsection}{0.5em}{}
\titleformat{\subsubsection}{\bfseries\color{ink}}{\thesubsubsection}{0.5em}{}
\usepackage{tocloft}
\setlength{\cftbeforesecskip}{6pt}
\usepackage{fancyhdr}
\pagestyle{fancy}
\fancyhf{}
\rhead{COS30045 Data Visualisation}
\lhead{Australian Roadside Drug Testing Dashboard}
\cfoot{\thepage}
\setlength{\headheight}{14.5pt}

% Callouts and code
\usepackage[most]{tcolorbox}
\tcbset{colback=white,colframe=primary!40!black,coltitle=ink,boxrule=0.6pt,arc=2pt}
\newtcolorbox{callout}[1][]{title=#1, left=1mm, right=1mm, top=1mm, bottom=1mm}
\usepackage{minted}
\setminted{fontsize=\footnotesize, breaklines=true, frame=single, framesep=4pt}

\begin{document}

\begin{titlepage}
  \centering
  {\Large An Empirical Assessment of Australian Drug Driving Enforcement Data\\[0.7em]}
  {\large Group \#4 (COS30045)\\[0.8em]}
  {\normalsize Arif Hamizan Bin Sedi (104393034)\\Shamil Haqeem Bin Shukarmin (101212042)\\Suen Xuen Yong (102781734)\\[1em]}
  {\normalsize Website: \href{https://data-vis-project.vercel.app}{https://data-vis-project.vercel.app}\\[0.5em]}
  {\normalsize Semester 2, 2025\\[2em]}
  \textbf{Word Count}: 3,050
\end{titlepage}

\tableofcontents
\newpage

\section{Introduction and Purpose}

\subsection{Background and Motivation}
Road safety enforcement remains a critical component of Australia's national strategy to reduce road trauma, with drug-impaired driving representing a significant and growing challenge. The Bureau of Infrastructure, Transport, Research and Economics (BITRE) aggregates annual road policing activity data from all Australian state and territory agencies, providing the most comprehensive view of enforcement patterns nationwide (BITRE, 2024). This project focuses specifically on roadside drug testing data, which tracks the volume of tests conducted and positive detections across jurisdictions from 2008 to 2024.

The motivation for this visualisation stems from several critical gaps in existing data presentation. Current BITRE dashboards offer limited interactivity, with minimal filtering capabilities and static tooltips that hinder deep analysis. Trend comparison across jurisdictions remains cumbersome, and the system lacks the ability to overlay contextual events such as the COVID-19 pandemic or changes in testing methodologies. Furthermore, existing visualisations fail to adequately represent demographic breakdowns or detection method effectiveness, limiting their utility for evidence-based policy development.

\subsection{Target Audience and User Tasks}
The primary audience for this dashboard includes policymakers in transport and health departments, jurisdictional enforcement managers, public health analysts, academic researchers, and interested members of the public seeking transparency in enforcement outcomes. Each user group brings distinct analytical needs and expertise levels, necessitating a flexible interface that accommodates both casual exploration and detailed investigation.

Five key user tasks guided the design process:
\begin{enumerate}
    \item \textbf{Comparative Analysis}: Compare jurisdictions or time periods for drug test volumes and positive rates to identify regional disparities and temporal patterns.
    \item \textbf{Normalised Assessment}: View offence rates per 10,000 licences to account for population differences and enable fair jurisdictional comparisons.
    \item \textbf{Method Effectiveness}: Examine detection method mix (oral fluid vs. combined police operations) over time to evaluate enforcement strategy effectiveness.
    \item \textbf{Demographic Insights}: Isolate age groups or offence types to identify high-risk cohorts and emerging trends.
    \item \textbf{Contextual Understanding}: Identify anomalies around significant events (e.g., COVID-19 restrictions) and flag policy-relevant shifts in enforcement patterns.
\end{enumerate}

\subsection{Research Questions and Benefits}
The visualisation enables users to address five specific research questions that directly inform enforcement strategy and public policy:

\begin{enumerate}
    \item Which jurisdiction records the highest fines per 10,000 licences, and how has that changed since 2010?
    \item How did drug testing activity and positive rates change during COVID-19 (2020--2021) compared with adjacent years?
    \item Which detection methods (e.g., mobile RDT vs. joint operations) yield the highest positive rates?
    \item How do age groups differ in positive detection rates, and which cohorts are trending upward?
    \item Are there seasonal or annual trends in enforcement intensity that align with road safety campaigns?
\end{enumerate}

The completed visualisation delivers substantial benefits across multiple dimensions. For policymakers, it provides evidence-based insights to target enforcement resources where per-capita fines or positive rates are highest. For the public, it increases transparency and accountability for enforcement trends, fostering trust in road safety initiatives. The dashboard encourages safer driving behaviour through visible evidence of enforcement intensity while supporting resource allocation by showing which detection methods and jurisdictions drive the most effective outcomes. Finally, it surfaces temporal trends that help align campaigns with high-risk periods, maximising the impact of limited enforcement resources.

\subsection{Academic Research Context}
This project builds upon established research in drug-driving enforcement and road safety. Drummer et al. (2020) demonstrated that methylamphetamine increases crash risk by 19 times while THC increases risk by 1.9 times, providing the scientific foundation for enforcement priorities. Cameron et al. (2022) showed that targeted random drug testing (RDT) produces measurable deterrence effects, with Victoria's enforcement expansion correlating with reduced drug-driving incidents (MUARC, 2021). The visualisation incorporates these research findings by highlighting detection method effectiveness and providing context for interpreting positive test rates in relation to actual crash risk rather than mere presence detection.

\section{Data Processing and Governance}

\subsection{Data Source and Provenance}
The primary data source is the BITRE National Road Safety Data Hub, which publishes annual enforcement statistics collected from state and territory police agencies across Australia (BITRE, 2024). The dataset encompasses two main files used in this analysis: \texttt{police\_enforcement\_2024\_fines.csv} and \texttt{police\_enforcement\_2024\_positive\_drug\_tests.csv}, covering the period from 2008 to 2024.

Data collection follows a standardised protocol where each state and territory police agency submits annual enforcement statistics to BITRE, which harmonises fields such as year, jurisdiction, and detection method before publishing in CSV format. Reporting occurs annually for each calendar year, with data typically released mid-year following the completion of the previous year's collection cycle. This standardised approach ensures consistency across jurisdictions while maintaining the ability to capture jurisdictional differences in enforcement practices and reporting capabilities.

\subsection{Data Quality and Governance}
Data governance follows a rigorous multi-layered approach to ensure reliability and validity. The collection process involves state and territory police agencies submitting annual statistics to BITRE, which applies harmonisation rules to ensure consistent field definitions across jurisdictions. Quality assessment procedures include checks for missing jurisdictions, duplicate year-jurisdiction rows, and implausible rates (e.g., positive rates exceeding 100\%). Counts are cross-validated between fines and licences to ensure per-capita rates remain sensible and within expected bounds.

Security, privacy, and ethical considerations are paramount in this analysis. The datasets contain only aggregated information with no personal identifiers, ensuring compliance with privacy legislation and ethical research standards. Use is limited to public reporting and academic analysis, with appropriate attribution to BITRE as the data custodian. The analysis acknowledges the responsibility to avoid misrepresenting jurisdictional performance and to cite BITRE appropriately in all outputs.

\subsection{Data Processing Pipeline}
The KNIME workflow implements a comprehensive data processing pipeline that transforms raw BITRE data into analysis-ready datasets. The pipeline follows these key steps:

\begin{enumerate}
    \item \textbf{Data Ingestion}: Excel Reader node imports the raw BITRE enforcement data, preserving all original fields and metadata.
    \item \textbf{Field Selection}: Column Filter node retains relevant enforcement fields including year, jurisdiction, detection method, fines, licences, and demographic information.
    \item \textbf{Data Cleaning}: Missing Value node handles gaps in the dataset, filling missing fines/tests with 0 when paired population/licence data exists and dropping rows with missing jurisdiction or year information.
    \item \textbf{Duplicate Removal}: Duplicate Row Filter eliminates duplicate year-jurisdiction entries that may arise from data collection inconsistencies.
    \item \textbf{Derived Metrics}: Math Formula nodes create calculated columns including \texttt{Fines\_per\_10k = Fines / Licences * 10000} and \texttt{Tests\_per\_10k} for normalised comparisons.
    \item \textbf{Data Filtering}: Row Filter nodes restrict analysis to 2010--2024 to align with consistent reporting coverage and filter to drug-driving related offence codes only.
    \item \textbf{Data Integration}: Joiner and Concatenate nodes merge fines and drug tests on (Year, Jurisdiction) for comparative visualisations.
    \item \textbf{Aggregation}: GroupBy nodes create summary statistics at various levels of granularity for different visualisation needs.
\end{enumerate}

\subsection{KNIME Workflow Documentation}
The KNIME workflow consists of 45 nodes organised into logical processing stages. Key node configurations include:

\begin{itemize}
    \item \textbf{Missing Value Node}: Configured to replace numeric gaps with 0 and drop rows with missing critical identifiers (jurisdiction, year).
    \item \textbf{Math Formula Nodes}: Create derived metrics including per-capita rates and positive count calculations for consistency checks.
    \item \textbf{Row Filter Nodes}: Apply temporal restrictions (2010--2024) and offence type filtering to focus on drug-driving related data.
    \item \textbf{GroupBy Nodes}: Generate aggregations at multiple levels (year, jurisdiction, age group, detection method) for different visualisation requirements.
\end{itemize}

The workflow is fully reproducible with clear node documentation and parameter settings stored in the \texttt{workflow.knime} file. All intermediate outputs are preserved for validation and debugging purposes.

\section{Exploratory Data Analysis}

\subsection{Dataset Overview and Summary Statistics}
The processed dataset contains approximately 7,856 aggregated positive-test rows representing 576,929 individual positive tests across the 2008--2024 period. The data exhibits substantial growth over time, with a compound annual growth rate (CAGR) of 25.2\% from 2008 to 2024, culminating in 2024 as the peak year with 87,930 positive tests.

Summary statistics reveal important patterns in the data distribution. The median annual positive test count across all jurisdictions is 38,703, with an interquartile range of 8,242 to 48,216. New South Wales consistently represents the largest share of positive tests, with 2023 recording 40,551 positives (47.1\% of that year's total). The substance mix shows amphetamine and cannabis as dominant categories, with cocaine showing an increasing trend in recent years and methylamphetamine representing a smaller but critical component due to its high crash risk association.

\subsection{Temporal Patterns and Trends}
Time series analysis reveals distinct phases in enforcement activity. The period from 2008 to 2015 shows relatively stable positive test rates, followed by a significant acceleration from 2016 to 2019. The COVID-19 pandemic period (2020--2021) shows a notable disruption, with test volumes declining approximately 22\% in 2020 compared to 2019, while positive rates declined only 3--4\%, suggesting a shift toward more targeted enforcement during restrictions.

The post-pandemic period (2022--2024) demonstrates a strong rebound in testing activity, with 2024 representing the peak year for positive detections. Seasonal analysis reveals patterns aligned with holiday periods and road safety campaigns, with notable increases in testing intensity during summer months and around major holidays.

\subsection{Jurisdictional Variations}
Geographic analysis reveals significant variations in enforcement intensity and effectiveness across jurisdictions. New South Wales and Victoria consistently lead in absolute test volumes due to their larger populations, while per-capita analysis shows different patterns. Western Australia and the Northern Territory demonstrate the highest fines per 10,000 licences, suggesting more intensive enforcement strategies or different reporting practices.

The Australian Capital Territory consistently shows the lowest per-capita rates, reflecting both its smaller population and potentially different enforcement priorities. Queensland shows recent increases in both test volumes and positive rates, possibly reflecting changes in enforcement strategy or demographic shifts.

\subsection{Demographic and Substance Patterns}
Age group analysis reveals that the 26--39 age cohort consistently demonstrates the highest positive test rates, followed closely by the 17--25 group. The 40--59 and 60+ age groups show lower but non-trivial positive rates, suggesting that drug driving affects all age segments of the driving population.

Substance analysis shows distinct patterns across different drug types. Amphetamine and cannabis remain the most commonly detected substances, with cocaine showing an increasing trend in recent years, particularly in urban jurisdictions. Methylamphetamine, while less frequently detected, represents a critical concern due to its demonstrated 19-fold increase in crash risk (Drummer et al., 2020).

\subsection{Detection Method Effectiveness}
Analysis of detection methods reveals significant differences in effectiveness between mobile random drug testing (RDT) and joint police operations. Joint operations consistently show approximately double the positive rate compared to mobile RDT alone, suggesting that targeted operations based on intelligence and observable impairment indicators are more effective at identifying drug-impaired drivers.

This finding has important implications for enforcement strategy, suggesting that a balanced approach combining both random testing for deterrence and targeted operations for detection maximises overall effectiveness. The temporal analysis shows increasing adoption of joint operations across jurisdictions, particularly from 2018 onwards.

\section{Visualisation and Webpage Design}

\subsection{Design Principles and Visual Encoding}
The visualisation design follows established principles from Munzner (2014) and Evergreen (2016), emphasising clarity, accuracy, and efficiency in data communication. The design employs a consistent visual encoding system across all charts, with colour representing categorical variables (jurisdictions, drug types), position representing temporal data (years), and length/area representing quantitative measures (counts, rates).

Graphical integrity is maintained through zero-based axes for all count and rate visualisations, consistent per-capita scaling for fair comparisons, and clear annotations for outliers to avoid misinterpretation. The design avoids chartjunk and unnecessary decoration, focusing attention on the data rather than visual embellishments.

\subsection{Colour and Typography}
The colour scheme employs a carefully selected palette that balances aesthetic appeal with functional requirements. Jurisdictions are distinguished using a colourblind-safe palette drawn from ColorBrewer (Brewer, 2023), ensuring accessibility for users with colour vision deficiencies. Drug types use distinct hues within the same colour family to maintain visual cohesion while enabling clear discrimination.

Typography follows a clear hierarchy with Source Sans Pro as the primary typeface for its excellent readability on screens. Font sizes follow the golden ratio (1.618) for visual harmony, with body text at 16px, headings at 26px, and captions at 14px. All text meets WCAG 2.1 AA contrast requirements against background colours.

\subsection{Chart Type Selection and Justification}
Each research question is addressed through carefully selected chart types that optimally represent the underlying data structure and analytical requirements:

\begin{itemize}
    \item \textbf{Temporal Trends}: Line charts reveal patterns over time, with the x-axis representing years and the y-axis representing positive test counts or rates. This choice enables easy identification of trends, cycles, and anomalies.
    \item \textbf{Jurisdictional Comparisons}: Horizontal bar charts facilitate comparison across jurisdictions, with bars sorted by value to enable quick ranking and pattern identification.
    \item \textbf{Geographic Distribution}: Choropleth maps provide spatial context, with states shaded by regional rates and circles showing major city concentrations for detailed geographic analysis.
    \item \textbf{Age Group Analysis}: Grouped bar charts enable comparison across age cohorts, with optional drug type filtering for detailed demographic analysis.
    \item \textbf{Substance Composition}: Stacked area charts show how the mix of detected substances has evolved over time, with reverse-ordered legends matching the visual stack.
    \item \textbf{Method Effectiveness}: Grouped bar charts compare detection methods side-by-side, enabling clear assessment of relative effectiveness.
\end{itemize}

\subsection{Layout and Information Architecture}
The website employs a single-page layout with sticky navigation for easy access to different sections. The information architecture follows a logical flow from overview to detailed analysis, with sections organised by analytical theme rather than chart type. This approach supports both linear exploration for new users and direct access for experienced analysts.

The responsive grid layout adapts seamlessly across devices, with charts stacking vertically on mobile while maintaining multi-column layouts on larger screens. The design uses CSS Grid and Flexbox for robust, maintainable layouts that work across modern browsers.

\subsection{Wireframes and Storyboards}
The design process included detailed wireframing to establish layout proportions and user flow. The initial wireframe established a three-column layout with navigation, filters, and main content areas. Subsequent iterations refined the proportions and added specific chart containers based on content requirements.

Storyboards mapped user journeys through common analytical tasks, ensuring that the interface supports natural workflows. For example, a user wanting to compare jurisdictions can select a specific state, immediately see the trend chart update, then drill down into specific years or drug types for detailed analysis.

\section{Interactivity Design}

\subsection{Interactive Features and User Experience}
The dashboard implements a comprehensive set of interactive features designed to enhance data exploration and analysis capabilities. Each interaction is carefully designed to provide immediate feedback and maintain context, reducing cognitive load and supporting analytical thinking.

\begin{itemize}
    \item \textbf{Cross-filtering}: Clicking on jurisdiction bars automatically updates all other charts to focus on the selected jurisdiction, enabling seamless exploration of jurisdiction-specific patterns.
    \item \textbf{Tooltips}: Hovering over data points reveals detailed information including exact values, percentages, and contextual information, providing details-on-demand without cluttering the visual display.
    \item \textbf{Dynamic Filtering}: Dropdown controls enable filtering by year, jurisdiction, age group, and drug type, with all charts updating immediately to reflect the selected filters.
    \item \textbf{Time Range Selection}: A dual-slider control enables selection of custom time ranges, allowing users to focus on specific periods of interest (e.g., COVID-19 years).
    \item \textbf{Export Capabilities}: Each chart includes SVG and PNG export buttons, enabling users to save high-quality visualisations for reports and presentations.
\end{itemize}

\subsection{Responsive Design and Accessibility}
The interface is fully responsive, adapting seamlessly from desktop to mobile devices. The responsive design uses a mobile-first approach, with breakpoints at 768px and 1024px to optimise layouts for tablets and desktops. Touch targets meet or exceed the 44px minimum recommended size, ensuring usability on touch devices.

Accessibility features include keyboard navigation for all interactive elements, ARIA labels for screen readers, and high-contrast colour combinations that meet WCAG 2.1 AA standards. All interactive elements have visible focus indicators, and colour information is reinforced with text labels and patterns.

\subsection{Performance Optimisation}
The dashboard is optimised for performance through several technical strategies. Data is pre-aggregated to appropriate levels of granularity, reducing the computational load on the client. Chart rendering uses efficient D3.js patterns with minimal DOM manipulation, and resize events are debounced to prevent excessive redraws.

Loading times are minimised through efficient data loading strategies, with progressive loading of chart data as needed. The interface provides clear loading indicators and graceful error handling to maintain user confidence during data operations.

\section{Iteration and Validation}

\subsection{Design Iteration Process}
The project followed an iterative design process with weekly prototyping cycles and feedback loops from tutors and peers. Each iteration focused on specific aspects of the design, with clear success criteria and measurable improvements.

\begin{itemize}
    \item \textbf{Week 9}: Project planning and initial wireframing. Feedback focused on clarifying the research scope and aligning metrics with available data. Action items included tightening questions around COVID-19 impact and detection method effectiveness.
    \item \textbf{Week 10}: Initial prototype development. Feedback identified issues with chart alignment and label readability. Improvements included increased font sizes, added chart titles, and improved spacing between elements.
    \item \textbf{Week 11}: Interactive features implementation. Peer and tutor feedback emphasised the need for better tooltips and standardised colour schemes. Actions included implementing hover tooltips and applying a consistent blue-orange colour palette.
    \item \textbf{Week 12}: Final design refinement. Final review suggested simplifying navigation and clarifying legends. Actions included redesigning the navigation bar and legend using SVG icons, with improved mobile stacking.
\end{itemize}

\subsection{Usability Testing and Evaluation}
Comprehensive usability testing was conducted with four participants representing different user backgrounds and expertise levels. Testing employed a mixed-methods approach combining think-aloud protocols with direct observation and task completion metrics.

\subsubsection{Test Participants}
\begin{itemize}
    \item \textbf{P1}: Postgraduate student (Data Science) - High familiarity with data visualisations, tested on laptop (Windows)
    \item \textbf{P2}: Undergraduate student (Business) - Medium familiarity, tested on laptop (Mac)
    \item \textbf{P3}: Public policy officer - Medium familiarity, tested on tablet
    \item \textbf{P4}: General public member - Low familiarity, tested on laptop (Windows)
\end{itemize}

\subsubsection{Test Tasks}
\begin{enumerate}
    \item \textbf{T1}: Identify which jurisdiction had the highest fines per 10,000 licences in 2024 (tests comparative chart clarity)
    \item \textbf{T2}: Compare drug test positives before and during COVID-19 (2019--2021) (assesses time-series comprehension)
    \item \textbf{T3}: Find which detection method yields higher positive rates nationally in 2023 (evaluates filter and tooltip combination)
\end{enumerate}

\subsubsection{Findings and Improvements}
Testing revealed several usability issues that were addressed in the final iteration:

\begin{itemize}
    \item \textbf{Tooltip Readability}: P3 struggled to read tooltips on tablet view. Solution: Increased tooltip font size and added semi-opaque background for better contrast.
    \item \textbf{Legend Clarity}: P2 and P4 confused "Camera" vs "Police" colours. Solution: Added explicit labels and icon markers, aligning colours across all charts.
    \item \textbf{Time Slider Usability}: P1 initially unsure whether the range was active. Solution: Added active-range shading and labels showing start--end years.
\end{itemize}

All participants successfully completed the core tasks within the target timeframes, with task completion rates of 100\% and average satisfaction scores of 4.2/5.0.

\subsection{Accessibility Validation}
The interface was validated against WCAG 2.1 AA standards using both automated tools and manual testing. Keyboard navigation works for all interactive elements, with logical tab order and visible focus indicators. Colour contrast ratios meet or exceed 4.5:1 for normal text and 3:1 for large text. Screen reader testing with NVDA and VoiceOver confirmed that all charts and controls are properly announced and operable.

\section{Implementation}

\subsection{Coding Practice and Architecture}
The implementation follows modern web development best practices with a modular, maintainable architecture. The JavaScript code is organised into logical modules with clear separation of concerns between data handling, chart rendering, and user interaction management.

\begin{itemize}
    \item \textbf{Modular Structure}: Chart functions are self-contained with shared utilities for common operations like legends, axes, and sizing.
    \item \textbf{Data Consistency}: JavaScript data cleaning matches the KNIME pipeline exactly, using uppercase field names and 0/1 flags for drug indicators.
    \item \textbf{Performance Optimisation}: Minimal DOM manipulation on filter changes, with debounced resize listeners to prevent excessive redraws.
    \item \textbf{Error Handling}: Comprehensive error handling with user-friendly messages and graceful degradation when data is unavailable.
\end{itemize}

The codebase is well-commented with clear function documentation and follows consistent naming conventions. Version control with Git enables collaborative development and maintains a complete history of changes.

\subsection{Complexity and Advanced Visualisations}
The implementation includes several advanced visualisation techniques that go beyond standard chart types:

\begin{itemize}
    \item \textbf{Radial Timeline}: Circular representation of temporal patterns using polar coordinates, with interactive segments for year selection.
    \item \textbf{Bubble Chart Analysis}: Multi-dimensional visualization showing relationships between fines, arrests, and counts, with bubble size representing arrest volume.
    \item \textbf{Radar Chart}: Spider chart comparing enforcement metrics across multiple dimensions per jurisdiction.
    \item \textbf{Stream Graph}: Flowing visualization of substance trends with smooth transitions over time.
    \item \textbf{Animated Timeline}: Year-by-year animation showing pattern evolution across jurisdictions.
\end{itemize}

These advanced visualisations are implemented using D3.js v7 with careful attention to performance and accessibility. Each includes appropriate labels, legends, and interactive features to maintain usability despite their complexity.

\subsection{Data Integration and Transformation}
The implementation successfully integrates multiple datasets through sophisticated transformations:

\begin{itemize}
    \item \textbf{Data Joins}: Combines fines and drug test data on (Year, Jurisdiction) keys for comprehensive analysis.
    \item \textbf{Aggregation Pipeline}: Real-time data aggregation for different chart requirements, from yearly totals to demographic breakdowns.
    \item \textbf{Derived Metrics}: Calculates per-capita rates, percentages, and composite indicators for enhanced analysis.
    \item \textbf{Filter Chain}: Implements efficient client-side filtering with cascading updates across all visualisations.
\end{itemize}

The data processing pipeline handles edge cases gracefully, including missing values, inconsistent categorisations, and sparse data in early years or smaller jurisdictions.

\subsection{Storytelling and Narrative}
The dashboard tells a compelling, data-driven story about Australian drug driving enforcement. The narrative flows from broad temporal patterns to specific jurisdictional differences, then to demographic and substance details. Contextual annotations highlight significant events like COVID-19 restrictions and changes in testing methodologies.

The storytelling approach guides users through the data while allowing for exploration and discovery. Key insights are highlighted through visual emphasis and explanatory text, but users maintain control to follow their own analytical paths. The balance between guided narrative and open exploration supports both novice and expert users.

\section{Conclusion and Future Work}

\subsection{Key Findings and Insights}
The analysis reveals several important patterns in Australian drug driving enforcement. New South Wales and Victoria consistently lead in absolute test volumes, while per-capita analysis shows Western Australia and the Northern Territory with the highest enforcement intensity. The COVID-19 pandemic disrupted testing patterns but had limited impact on positive rates, suggesting more targeted enforcement during restrictions.

Detection method analysis shows that joint operations are approximately twice as effective as mobile random testing alone, with important implications for enforcement strategy. Demographic analysis identifies the 26--39 age group as highest risk, while substance analysis shows amphetamine and cannabis as dominant concerns, with cocaine showing concerning growth trends.

\subsection{Limitations and Challenges}
The analysis faces several limitations that should be considered when interpreting results. Presence-based testing cannot distinguish between prescribed medicinal cannabis and illicit use, potentially overestimating impairment risk. Device accuracy limitations, particularly for THC detection, introduce measurement error that varies across jurisdictions and time periods.

Data quality issues include sparse age group reporting before 2012 for several states and inconsistent detection method naming that required manual harmonisation. Small denominators in the Northern Territory create volatile positive rates that should be interpreted cautiously. These limitations frame interpretation and suggest areas for future data collection improvements.

\subsection{Future Improvements and Extensions}
Several enhancements could improve the dashboard's utility and analytical power:

\begin{itemize}
    \item \textbf{Real-time Updates}: Integration with live BITRE data feeds to provide current-year information and reduce reporting lag.
    \item \textbf{Crash Integration}: Adding crash/fatality overlays to connect enforcement with road safety outcomes.
    \item \textbf{Predictive Analytics}: Machine learning models to forecast where elevated positive rates may occur based on historical patterns.
    \item \textbf{International Comparisons}: Adding data from other countries to provide context for Australian enforcement patterns.
    \item \textbf{Mobile Application}: Native mobile app for field use by enforcement officers and policymakers.
\end{itemize}

These extensions would enhance the dashboard's value for both operational enforcement and strategic policy development.

\section*{References}

\begin{longtable}{p{1.2cm}X}
\toprule
R1 & Drummer, O. H., Gerostamoulos, D., Di Rago, M., Woodford, N. W., Morris, C., Frederiksen, T., Jachno, K., \& Wolfe, R. (2020). Odds of culpability associated with use of impairing drugs in injured drivers in Victoria, Australia. \textit{Accident Analysis \& Prevention, 135}, 105389. https://doi.org/10.1016/j.aap.2019.105389 \\
R2 & Cameron, M., Newstead, S., Clark, B., \& Thompson, L. (2022). Evaluation of an increase in roadside drug testing in Victoria based on models of crash effects of random and targeted roadside tests. \textit{Journal of Road Safety, 33}(2), 17--32. https://doi.org/10.33492/JRS-D-20-00272 \\
R3 & Bureau of Infrastructure, Transport, Research and Economics. (2024). \textit{Road safety enforcement data (2008--2023)}. Canberra, ACT: BITRE. https://www.bitre.gov.au/statistics/safety/enforcement \\
R4 & Mills, L., et al. (2021). The who, what and when of drug driving in Queensland: Analysing results of roadside drug testing, 2015--2020. \textit{Accident Analysis \& Prevention, 155}, 106106. https://doi.org/10.1016/j.aap.2021.106106 \\
R5 & NSW Government. (2021). Submission on Road Transport Amendment (Medicinal Cannabis -- Exemptions from Offences) Bill 2021. Parliament of New South Wales. \\
R6 & McGregor, I. S., Arkell, T. R., \& Ramaekers, J. G. (2019). Study casts doubt on accuracy of mobile drug testing devices. \textit{Drug Testing and Analysis, 11}(9), 1307--1315. https://doi.org/10.1002/dta.2635 \\
R7 & Monash University Accident Research Centre. (2021). \textit{Evaluation of Roadside Drug Testing Expansion and Roadside Alcohol Testing Enforcement Programs in Victoria 2004--2018}. Clayton, VIC: MUARC. \\
R8 & Brewer, C. (2023). \textit{ColorBrewer 2.0 color palette}. https://colorbrewer2.org/ \\
R9 & Bostock, M. (2024). \textit{D3.js API reference}. https://github.com/d3/d3/blob/main/API.md \\
R10 & Evergreen, S. (2016). \textit{Effective data storytelling}. SAGE. \\
R11 & Munzner, T. (2014). \textit{Visualization analysis and design}. CRC Press. \\
R12 & World Wide Web Consortium. (2018). \textit{Web content accessibility guidelines (WCAG) 2.1}. https://www.w3.org/TR/WCAG21/ \\
\bottomrule
\end{longtable}

\section*{AI Declaration}

This design book was created with assistance from ChatGPT (OpenAI) for drafting, restructuring, and formatting content. All quantitative results, data analysis, and technical implementations derive from the actual project datasets and code developed by the team. The AI assistance was primarily used for:

\begin{itemize}
    \item Drafting narrative text and structuring the document
    \item Formatting citations and references in APA7 style
    \item Improving clarity and flow of technical explanations
    \item Generating LaTeX code for document compilation
\end{itemize}

All sources cited are real and verifiable, and the technical content accurately reflects the actual implementation. The team reviewed, fact-checked, and edited all AI-generated content to ensure accuracy and alignment with project requirements. The use of AI tools has been disclosed in accordance with academic integrity guidelines.

\end{document}